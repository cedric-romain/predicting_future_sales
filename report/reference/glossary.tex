% \gls{ } : This command prints the term in lowercase characters. For example, \gls{glsy} will print glossary.
% \Gls{ } : This is similar to \gls{ } command. The only difference is it will print the first character in the uppercase For example, \Gls{glsy} will print Glossary.
% \glspl{ } : This command is similar to \gls{ }. The only difference is that it will convert the term into its plural form.
% \Glspl{ } : This command is similar to \Gls{ } command, with the difference that it will convert the term in its plural also.

\newglossaryentry{df}
{
    name=DataFrame,
    description={Datatype from the Pandas library. Two-dimensional, size-mutable, potentially heterogeneous tabular data.... \href{https://pandas.pydata.org/pandas-docs/stable/reference/api/pandas.DataFrame.html}{Link to the full documentation.}}
}
\newglossaryentry{scikit}
{
    name=scikit-learn,
    description={Scikit-learn is an open source machine learning library that supports supervised and unsupervised learning... \href{https://scikit-learn.org/stable/getting_started.html}{Link to the full documentation.}}
}
\newglossaryentry{futuresales}
{
    name=Future Sales,
    description={Name of the dataset from the Kaggle competition. \href{https://www.kaggle.com/c/competitive-data-science-predict-future-sales/}{Link.}}
}
\newglossaryentry{hyperparameter}
{
    name=hyperparameter,
    description={Parameter that is set before the learning process. These can be tuned according to the statistical models that are employed and has a direct impact on the performance.}
}
