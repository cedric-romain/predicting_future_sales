\subsection{Create zero sales}

The training dataset tells us exactly which item got sold when and where. At the same time, this tells us exactly which shop did not sell any given item on a given date.
With this information, we can craft the possibly most important feature of them all: add the records from a possible combination of a given month, all shops and all items and fill the sale record with 0. This operation was dubbed \enquote{create zero sales}.
The mathematical function of this operation can be expressed with the following cartesian product:

\vspace*{-4mm}
$$
\{\text{\texttt{date\_block\_num}}_i\}
\quad \times \quad
\{{\text{\texttt{shop\_id}}}\}
\quad  \times \quad 
\{{\text{\texttt{item\_id}}}\}
\quad \mid \quad
i \cap [0; 33]
$$

This operation was implemented using the Python function \texttt{product} from the \texttt{itertools} package from the Python standard library.\footnote{\href{https://docs.python.org/3/library/itertools.html\#itertools.product}{\texttt{itertools} - official documentation}}

\noindent \textit{N.B. that this ignores the months which have neither sold anything in a given month.}

\subsection{Shop information}

Foo

\subsection{Category information}

Foo

\subsection{Lag features}

Foo

\subsection{Average prices}

Foo

\subsection{Additional features}

Foo
