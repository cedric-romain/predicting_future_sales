% SOURCE: https://www.latextemplates.com/template/science-journal

% Use only LaTeX2e, calling the article.cls class and 12-point type.

\ifdigitalcopy
\documentclass[a4paper,12pt]{article}
\else
\documentclass[a4paper,12pt,twoside]{article}
\fi

% Users of the {thebibliography} environment or BibTeX should use the
% scicite.sty package, downloadable from *Science* at
% www.sciencemag.org/about/authors/prep/TeX_help/ .
% This package should properly format in-text
% reference calls and reference-list numbers.

\usepackage{external_content/layout/scicite}

% Use times if you have the font installed; otherwise, comment out the
% following line.

\usepackage{times}

% The preamble here sets up a lot of new/revised commands and
% environments.  It's annoying, but please do *not* try to strip these
% out into a separate .sty file (which could lead to the loss of some
% information when we convert the file to other formats).  Instead, keep
% them in the preamble of your main LaTeX source file.


% The following parameters seem to provide a reasonable page setup.

\usepackage{geometry}
 \geometry{
 a4paper,
 left=24mm,
 right=24mm,
 top=33mm,
 bottom=33mm
 }
 % \oddsidemargin 2mm


%The next command sets up an environment for the abstract to your paper.

\newenvironment{sciabstract}{%
\begin{quote} \bf}
{\end{quote}}


% The following lines set up an environment for the last note in the
% reference list, which commonly includes acknowledgments of funding,
% help, etc.  It's intended for users of BibTeX or the {thebibliography}
% environment.  Users who are hand-coding their references at the end
% using a list environment such as {enumerate} can simply add another
% item at the end, and it will be numbered automatically.

\newcounter{lastnote}
\newenvironment{scilastnote}{%
\setcounter{lastnote}{\value{enumiv}}%
\addtocounter{lastnote}{+1}%
\begin{list}%
{\arabic{lastnote}.}
{\setlength{\leftmargin}{.22in}}
{\setlength{\labelsep}{.5em}}}
{\end{list}}

% enable links
\usepackage{hyperref}

% glossary
\usepackage[automake,acronym]{glossaries}
\loadglsentries{reference/glossary.tex}
\loadglsentries{reference/acronym.tex}
\makeglossaries

% add media
\usepackage{graphicx}

% add code listings
\usepackage{listings}

% add colours
\usepackage{xcolor}

% for tables
\usepackage{wrapfig}

% enhance captions
\usepackage{caption}
\usepackage{subcaption}



% bibliography
\usepackage[sorting=none]{biblatex}
\bibliography{reference/scibib}

% add todo notes
\usepackage{todonotes}
\setlength {\marginparwidth}{20mm}  % distance from margin


% control the spacing between lines
\usepackage{setspace}
\onehalfspacing

% for \displayquote
\usepackage{csquotes}

% avoid widow and orphan lines in document
% https://en.wikipedia.org/wiki/Widows_and_orphans
\clubpenalty=10000
\widowpenalty=10000

% control ToC formatting
\usepackage{tocloft}

% boxes around text
\usepackage{tcolorbox}


% SVG support, N.B. this requires `inkscape` to be installed and present in PATH
% as well as the option `--shell-escape` to be enabled.
% using sublime, append this to LaTeXing settings: 
%     "build_arguments": [
%         "--shell-escape"
%     ],
\usepackage{svg}

% fine tune typography
\usepackage{microtype}

% math packages
\usepackage{amsmath}
\usepackage{amsfonts}
\usepackage{amssymb}

% enhanced footnotes
\usepackage[bottom]{footmisc}
